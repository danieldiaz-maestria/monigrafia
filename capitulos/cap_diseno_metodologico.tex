\chapter{DISEÑO METODOLÓGICO}
\label{ch:metodologia_experimental}

En este capítulo se describe detalladamente la metodología empleada para llevar a cabo la investigación experimental. Se presenta el enfoque metodológico general, el diseño del sistema de medición, la configuración del escenario experimental, los protocolos de recolección de datos y las técnicas de procesamiento y análisis que permitirán responder a las preguntas de investigación planteadas.

\section{CONFIGURACIÓN DEL HARDWARE}
\label{sec:configuracion_hardware}

\subsection{Dispositivos UWB}

\subsubsection{Selección de Dispositivos}

Para este estudio se han seleccionado módulos de desarrollo Qorvo DWM1001, que integran un transceptor UWB DW1000, un microcontrolador Nordic Semiconductor nRF52832 y un sensor de movimiento. Los módulos operan con el firmware PANS (Positioning and Networking Stack) de Qorvo, el cual implementa la técnica TWR para estimación de distancia. Estos dispositivos están configurados para operar en la banda de frecuencia central de 6.5 GHz. La elección de estos dispositivos se fundamenta en:

\begin{itemize}
\item Capacidad de operar en la banda de 6.5 GHz con el ancho de banda necesario para lograr resolución temporal adecuada.
\item Soporte nativo para mediciones TWR con registro de timestamps de alta precisión.
\item Interfaz de comunicación accesible que permite la extracción de datos crudos de ToF y otras métricas de señal.
\item Consumo energético compatible con operación portátil prolongada.
\item Disponibilidad de documentación técnica detallada y herramientas de desarrollo.
\end{itemize}

\subsubsection{Parámetros de Configuración}

Los dispositivos UWB se configuran con los siguientes parámetros operacionales:

\begin{itemize}
\item \textbf{Frecuencia central:} 6489.6 MHz (Canal 5)
\item \textbf{Ancho de banda:} 499.2 MHz
\item \textbf{Tasa de transmisión de datos:} 6.8 Mbps
\item \textbf{Frecuencia de repetición de pulsos (PRF):} 64 MHz
\item \textbf{Longitud del preámbulo:} 128 símbolos
\item \textbf{Código del preámbulo (TX/RX):} 10
\item \textbf{Potencia de transmisión:} -17 dBm
\item \textbf{Sensibilidad del receptor:} -93 dBm
\end{itemize}

Estos parámetros se seleccionan para optimizar el compromiso entre exactitud de medición, alcance efectivo y consumo energético, siguiendo las recomendaciones del fabricante y las mejores prácticas identificadas en la revisión de literatura.

\subsection{Sistema de Referencia de Posición}

[COMPLETAR: Describir si se utiliza un sistema óptico de captura de movimiento, marcadores fiduciales, o un método manual de posicionamiento con puntos de referencia marcados en el piso]


\section{DISEÑO DEL ESCENARIO EXPERIMENTAL}
\label{sec:escenario_experimental}

\subsection{Selección y Características del Entorno}

El experimento se desarrolla en [ESPECIFICAR UBICACIÓN EXACTA, ej. "el Salón 305 del Edificio de Ingeniería Electrónica y Telecomunicaciones de la Universidad del Cauca"]. Este espacio ha sido seleccionado por presentar características representativas de escenarios de interiores típicos donde se implementan sistemas de posicionamiento:

\begin{itemize}
\item \textbf{Dimensiones:} 20 m de largo × 3 m de ancho (escenario tipo corredor)
\item \textbf{Materiales constructivos:} Paredes de concreto, piso de baldosa cerámica, techo de placa de concreto
\item \textbf{Mobiliario:} Corredor típico de edificio universitario con mobiliario mínimo
\item \textbf{Fuentes de interferencia:} Redes WiFi institucionales activas en bandas de 2.4 GHz y 5 GHz, equipos electrónicos de oficinas adyacentes
\end{itemize}

\subsection{Disposición de Nodos Ancla}

Se despliegan cuatro nodos ancla fijos en configuración [ESPECIFICAR: rectangular, cuadrada, etc.] que define el área de cobertura del sistema de posicionamiento. Las coordenadas de instalación de los nodos ancla son:

\begin{itemize}
\item \textbf{Ancla 1:} $(x_1, y_1$ = [ESPECIFICAR coordenadas en metros]
\item \textbf{Ancla 2:} $(x_2, y_2)$ = [ESPECIFICAR coordenadas en metros]
\item \textbf{Ancla 3:} $(x_3, y_3)$ = [ESPECIFICAR coordenadas en metros]
\item \textbf{Ancla 4:} $(x_4, y_4)$ = [ESPECIFICAR coordenadas en metros]
\end{itemize}

Los nodos ancla se montan a una altura de 1.5 metros sobre el nivel del piso, utilizando trípodes, asegurando estabilidad mecánica durante todo el período experimental.

La geometría de despliegue se diseña para minimizar la dilución de precisión geométrica (GDOP) en el área de interés, maximizando así la exactitud de localización en condiciones de LOS.


\section{PROTOCOLO DE RECOLECCIÓN DE DATOS}
\label{sec:protocolo_recoleccion}

\subsection{Población y Muestra}

Para evaluar la variabilidad del efecto de obstrucción corporal entre individuos con diferentes características antropométricas, el estudio incluye la participación de [ESPECIFICAR número] voluntarios. Los criterios de selección consideran diversidad en:

\begin{itemize}
\item \textbf{Estatura:} Rango de [ESPECIFICAR] a [ESPECIFICAR] cm
\item \textbf{Peso:} Rango de [ESPECIFICAR] a [ESPECIFICAR] kg
\item \textbf{Índice de Masa Corporal:} Cubriendo categorías [ESPECIFICAR]
\item \textbf{Género:} [ESPECIFICAR distribución]
\end{itemize}

Cada participante firma un consentimiento informado donde se explica el propósito del estudio, los procedimientos a realizar, y el uso de los datos recolectados, siguiendo los lineamientos éticos establecidos por [ESPECIFICAR comité de ética o normativa aplicable].

\subsection{Posiciones del Dispositivo en el Cuerpo}

Se evalúan múltiples ubicaciones de portación del nodo móvil UWB, seleccionadas por su relevancia en aplicaciones prácticas de seguimiento de personas:

\begin{enumerate}
\item \textbf{Frente (Pecho):} Dispositivo colocado en el centro del pecho, aproximadamente a la altura del esternón, orientado hacia adelante.
\item \textbf{Espalda:} Dispositivo colocado en el centro de la espalda, entre los omóplatos, orientado hacia atrás.
\item \textbf{Cintura (Cadera):} Dispositivo ubicado en la cintura, sobre la cadera derecha, orientado lateralmente.
\item \textbf{Muñeca:} Dispositivo colocado en la muñeca izquierda, simulando un dispositivo tipo smartwatch.
\item \textbf{Tobillo:} Dispositivo colocado en el tobillo derecho, orientado lateralmente.
\end{enumerate}

Para cada ubicación se registra la altura exacta sobre el nivel del piso y la orientación del dispositivo mediante fotografías y anotaciones en el protocolo experimental.

\subsection{Trayectorias y Posiciones de Medición}

\subsubsection{Mediciones Estáticas}

En una primera fase, se realizan mediciones estáticas donde el participante permanece inmóvil en posiciones específicas dentro del escenario. Se define una malla de puntos de referencia con separación de 1 metro, cubriendo un rango de distancias desde 1 m hasta 13 m entre el nodo móvil y el nodo fijo.

En cada posición de la malla, el participante adopta 2 orientaciones diferentes (LOS y NLOS), permitiendo evaluar el efecto de la obstrucción corporal de manera sistemática. En la condición LOS, el participante se orienta de manera que la parte del cuerpo que porta el dispositivo mantiene visibilidad directa con el nodo de referencia. En la condición NLOS, el participante se orienta de espaldas al nodo fijo.

Para cada combinación de posición-orientación-ubicación del dispositivo, se registran 250 mediciones de distancia estimada, asegurando un análisis estadístico robusto.

\subsubsection{Mediciones Dinámicas}

[ESPECIFICAR si se incluyen: En una segunda fase, se realizan mediciones dinámicas donde el participante se desplaza siguiendo trayectorias predefinidas...]

\subsection{Condiciones de Control}

Para garantizar la reproducibilidad del experimento y minimizar la variabilidad no controlada:

\begin{itemize}
\item Las mediciones se realizan en horarios donde la actividad en las áreas circundantes es mínima, reduciendo interferencia de personas transitando.
\item Se verifica antes de cada sesión que no haya cambios significativos en el entorno (mobiliario movido, nuevos objetos metálicos, etc.).
\item Se registran las condiciones ambientales: temperatura, humedad relativa, presencia de dispositivos electrónicos activos.
\item Se realiza una medición de calibración en LOS (sin participante presente) al inicio de cada sesión experimental para verificar la estabilidad del sistema.
\end{itemize}

\subsection{Procedimiento de Medición}

El protocolo detallado para cada sesión experimental es el siguiente:

\begin{enumerate}
\item \textbf{Preparación:} Encendido de los nodos ancla, verificación de comunicación, sincronización de relojes del sistema, inicialización del software de adquisición de datos.

\item \textbf{Calibración LOS:} Mediciones de referencia en ausencia de obstrucción, colocando el nodo móvil en posiciones conocidas para verificar la exactitud del sistema.

\item \textbf{Instrumentación del Participante:} Colocación del dispositivo UWB en la ubicación corporal correspondiente a la sesión actual, verificación de la correcta sujeción y orientación.

\item \textbf{Adquisición de Datos:} El participante se posiciona en cada punto de la malla según las instrucciones del operador. En cada posición, adopta secuencialmente las orientaciones especificadas. El sistema registra automáticamente las mediciones de ToF hacia los cuatro nodos ancla. El operador anota en la planilla experimental cualquier observación relevante.

\item \textbf{Verificación de Integridad de Datos:} Al finalizar cada conjunto de mediciones, se verifica que no haya pérdidas de paquetes o datos corruptos. Si se detectan anomalías, se repite el conjunto afectado.

\item \textbf{Descarga y Respaldo de Datos:} Los datos se descargan del sistema de adquisición, se organizan en la estructura de directorios definida y se realiza un respaldo en ubicaciones redundantes.
\end{enumerate}


\section{PROCESAMIENTO Y ANÁLISIS DE DATOS}
\label{sec:procesamiento_datos}

\subsection{Preprocesamiento de Datos Crudos}

Los datos crudos registrados por el sistema UWB incluyen timestamps de alta resolución, mediciones de ToF, estimaciones de potencia de señal recibida y otros parámetros de diagnóstico. El preprocesamiento comprende:

\begin{itemize}
\item \textbf{Filtrado de valores atípicos:} Detección y remoción de mediciones claramente erróneas (ej. ToF negativo, distancias fuera del rango físico posible) mediante criterios estadísticos robustos.

\item \textbf{Conversión de ToF a Distancia:} Aplicación de la ecuación $d = \text{ToF} \times c$, donde $c$ es la velocidad de la luz.

\item \textbf{Corrección de sesgos sistemáticos:} Si se identifican sesgos consistentes en las mediciones de calibración LOS, se aplican correcciones de offset para cada nodo ancla.

\item \textbf{Sincronización temporal:} Alineación temporal de las mediciones desde los cuatro nodos ancla cuando corresponden a la misma posición instantánea del participante.
\end{itemize}

\subsection{Cálculo de Métricas de Error}

Para cada medición de distancia entre el nodo móvil y un nodo ancla, se calcula:

\begin{equation}
e_i = d_{\text{medida},i} - d_{\text{real},i}
\end{equation}

donde $d_{\text{real},i}$ es la distancia euclidiana verdadera calculada a partir de las coordenadas conocidas del nodo ancla $i$ y la posición de referencia del participante.

Se definen las siguientes métricas estadísticas:

\begin{itemize}

\item \textbf{Error Absoluto Medio (Mean Absolute Error, MAE):}
\begin{equation}
\text{MAE} = \frac{1}{N} \sum_{j=1}^{N} |e_j|
\end{equation}

\item \textbf{Raíz del Error Cuadrático Medio (Root Mean Square Error, RMSE):}
\begin{equation}
\text{RMSE} = \sqrt{\frac{1}{N} \sum_{j=1}^{N} e_j^2}
\end{equation}


\item \textbf{Percentiles:} Se calculan los percentiles 50 (mediana), 90 y 95 del error absoluto para caracterizar la distribución completa del error.
\end{itemize}

\subsection{Análisis Estadístico}

\subsubsection{Caracterización de Distribuciones}

Para cada condición experimental (combinación de ubicación corporal, orientación del cuerpo, distancia), se ajustan distribuciones de probabilidad teóricas a los errores observados. Se evalúan distribuciones:

\begin{itemize}
\item Normal (Gaussiana)
\item Log-normal
\item Gamma
\end{itemize}

La bondad de ajuste se evalúa mediante pruebas de Kolmogorov-Smirnov y criterios de información (AIC, BIC).

\subsubsection{Análisis de Varianza (ANOVA)}

Se emplea ANOVA multifactorial para determinar qué factores experimentales tienen efecto estadísticamente significativo sobre el error de distancia:

\begin{itemize}
\item Factor 1: Ubicación del dispositivo en el cuerpo (5 niveles)
\item Factor 3: Distancia del nodo móvil a cada ancla (variable continua o discretizada)
\item Factor 4: Características antropométricas del participante (covariables)
\end{itemize}

Las interacciones entre factores también se analizan para identificar si, por ejemplo, el efecto de la orientación del cuerpo depende de la ubicación corporal del dispositivo.

\subsubsection{Análisis de Correlación}

Se calcula el coeficiente de correlación de Pearson entre el error de distancia y variables continuas como:
\begin{itemize}
\item Distancia verdadera nodo móvil - nodo ancla
\item Ángulo de orientación del cuerpo
\item Estatura del participante
\item Peso del participante
\item Potencia de señal recibida (RSS)
\end{itemize}

\subsection{Estimación de Posición y Evaluación del Sistema Completo}

A partir de las mediciones de distancia (con error introducido por BS) hacia los cuatro nodos ancla, se estima la posición 2D del nodo móvil mediante algoritmos de trilateración:

\subsubsection{Trilateración por Mínimos Cuadrados}

Se plantea el sistema de ecuaciones:
\begin{equation}
\sqrt{(x - x_i)^2 + (y - y_i)^2} = d_{\text{medida},i} \quad \text{para } i = 1, 2, 3, 4
\end{equation}

donde $(x, y)$ es la posición desconocida del nodo móvil y $(x_i, y_i)$ son las coordenadas conocidas de los nodos ancla. Este sistema no lineal se linealiza y resuelve mediante el método de mínimos cuadrados ponderados.

\subsubsection{Filtro de Kalman}

[ESPECIFICAR si se implementa: Se implementa un Filtro de Kalman para suavizar las estimaciones de posición a lo largo del tiempo...]

\subsubsection{Métricas de Error de Posición}

Para la posición estimada $(\hat{x}, \hat{y})$ respecto a la posición verdadera $(x_{\text{real}}, y_{\text{real}})$, se calculan:

\begin{itemize}
\item \textbf{Error de Posición 2D:}
\begin{equation}
e_{\text{pos}} = \sqrt{(\hat{x} - x_{\text{real}})^2 + (\hat{y} - y_{\text{real}})^2}
\end{equation}

\item \textbf{Error de posición medio, mediana, RMSE, percentiles} siguiendo las mismas definiciones que para el error de distancia.

\item \textbf{Error Circular Probable (CEP):} Radio del círculo centrado en la posición verdadera que contiene el 50\% de las estimaciones de posición.

\item \textbf{Exactitud al 95\%:} Radio del círculo que contiene el 95\% de las estimaciones.
\end{itemize}


\section{HERRAMIENTAS DE SOFTWARE}
\label{sec:herramientas_software}

\subsection{Software de Adquisición de Datos}

[ESPECIFICAR: Lenguaje de programación y bibliotecas utilizadas para la interfaz con los dispositivos UWB, ej. Python con biblioteca DWM1001-API, C++ con SDK del fabricante, etc.]

\subsection{Software de Procesamiento y Análisis}

El procesamiento de datos y análisis estadístico se realiza utilizando:

\begin{itemize}
\item \textbf{Python 3.x} con las bibliotecas científicas:
    \begin{itemize}
    \item \texttt{NumPy}: Operaciones numéricas y álgebra lineal
    \item \texttt{Pandas}: Manipulación y análisis de datos estructurados
    \item \texttt{SciPy}: Funciones estadísticas avanzadas, ajuste de distribuciones, ANOVA
    \item \texttt{Matplotlib} y \texttt{Seaborn}: Visualización de datos
    \item \texttt{scikit-learn}: Implementación de filtros y algoritmos de localización
    \end{itemize}
\item [ESPECIFICAR alternativas si aplica: MATLAB, R, etc.]
\end{itemize}

\subsection{Control de Versiones y Reproducibilidad}

Todo el código desarrollado se gestiona mediante Git y se almacena en un repositorio [ESPECIFICAR: privado/público, plataforma]. Los datos experimentales se organizan en formato estándar [ESPECIFICAR: CSV, HDF5, etc.] con metadatos descriptivos. Se proporciona un entorno computacional reproducible mediante [ESPECIFICAR: archivo requirements.txt, contenedor Docker, etc.] que especifica las versiones exactas de todas las dependencias de software.


\section{CONSIDERACIONES ÉTICAS}
\label{sec:consideraciones_eticas}

El protocolo experimental ha sido [ESPECIFICAR: "aprobado por el Comité de Ética de..." o "diseñado siguiendo los lineamientos éticos de..."]. Los participantes:

\begin{itemize}
\item Son informados detalladamente sobre el propósito del estudio y los procedimientos.
\item Firman un consentimiento informado antes de participar.
\item Pueden retirarse del estudio en cualquier momento sin consecuencias.
\item Sus datos personales (nombre, edad, etc.) son anonimizados y protegidos según normativas de protección de datos [ESPECIFICAR normativa aplicable: Ley 1581 de 2012 en Colombia, GDPR en Europa, etc.].
\item No se someten a ningún procedimiento invasivo o de riesgo para su salud o integridad física.
\end{itemize}


\section{RESUMEN DEL CAPÍTULO}

Este capítulo ha presentado el diseño metodológico completo de la investigación experimental, estructurada en dos fases complementarias:

\textbf{Fase 1 - Validación del Sistema de Estimación de Distancia:} Evaluación exhaustiva de la exactitud en la medición de distancias UWB bajo diferentes ubicaciones corporales del dispositivo móvil y condiciones de propagación (LOS/NLOS) en dos escenarios representativos. Esta fase establece la línea base de desempeño del sistema y caracteriza el impacto de la obstrucción corporal sobre las mediciones de distancia.

\textbf{Fase 2 - Evaluación del Sistema de Posicionamiento 2D Completo:} Implementación y evaluación del sistema completo de localización indoor utilizando cuatro nodos ancla en configuración geométrica optimizada. Esta fase integrará las mediciones de distancia hacia múltiples anclas para estimar posiciones 2D mediante algoritmos de trilateración y filtrado, evaluando el error de posicionamiento resultante bajo las mismas condiciones experimentales de la Fase 1.

La metodología propuesta, fundamentada en el Modelo en V, asegura que cada aspecto del sistema experimental sea verificado y validado antes de la recolección de datos definitiva, minimizando el riesgo de errores sistemáticos y maximizando la calidad y reproducibilidad de los resultados. El siguiente capítulo presentará los resultados obtenidos en la Fase 1 de validación.
